% Этот шаблон документа разработан в 2014 году
% Данилом Фёдоровых (danil@fedorovykh.ru) 
% для использования в курсе 
% <<Документы и презентации в \LaTeX>>, записанном НИУ ВШЭ
% для Coursera.org: http://coursera.org/course/latex .
% Исходная версия шаблона --- 
% https://www.writelatex.com/coursera/latex/2

\documentclass[../main.tex]{subfiles}

%%% Работа с русским языком
\usepackage{cmap}					% поиск в PDF
\usepackage{mathtext} 				% русские буквы в формулах
\usepackage[T2A]{fontenc}			% кодировка
\usepackage[utf8]{inputenc}			% кодировка исходного текста
\usepackage[english,russian]{babel}	% локализация и переносы

%%% Дополнительная работа с математикой
\usepackage{amsfonts,amssymb,amsthm,mathtools} % AMS
\usepackage{amsmath}
\usepackage{icomma} % "Умная" запятая: $0,2$ --- число, $0, 2$ --- перечисление

%% Номера формул
%\mathtoolsset{showonlyrefs=true} % Показывать номера только у тех формул, на которые есть \eqref{} в тексте.
%% Шрифты
\usepackage{euscript}	 % Шрифт Евклид
\usepackage{mathrsfs} % Красивый матшрифт

%% Свои команды
\DeclareMathOperator{\sgn}{\mathop{sgn}}

%% Перенос знаков в формулах (по Львовскому)
\newcommand*{\hm}[1]{#1\nobreak\discretionary{}
	{\hbox{$\mathsurround=0pt #1$}}{}}

%%% Работа с картинками
\usepackage{graphicx}  % Для вставки рисунков
\graphicspath{{images/1/}{images/2/}{{images/4/}}}  % папки с картинками
\setlength\fboxsep{3pt} % Отступ рамки \fbox{} от рисунка
\setlength\fboxrule{1pt} % Толщина линий рамки \fbox{}
\usepackage{wrapfig} % Обтекание рисунков и таблиц текстом

%%% Работа с таблицами
\usepackage{array,tabularx,tabulary,booktabs} % Дополнительная работа с таблицами
\usepackage{longtable}  % Длинные таблицы
\usepackage{multirow} % Слияние строк в таблице


\usepackage{indentfirst}
\usepackage{hyperref}
\usepackage{booktabs}
\usepackage{float}
\usepackage[table]{xcolor}

% код в matlab
\usepackage{matlab-prettifier}
\usepackage{listings,lstautogobble}
\lstset{autogobble=true}



\begin{document} % конец преамбулы, начало документа
	
	В данной главе будет реализована одна вариация генетического алгоритма, которая затем будет протестирована на нескольких задачах. 
	
	\subsection{Описание генетического алгоритма}
	
	Для начала кратко опишем генетический алгоритм: данная процедура используется для решения задачи минимизации некоторой функции $f: \mathbb{R}^n \rightarrow \mathbb{R}$ на некотором множестве $D \subseteq \mathbb{R}^n$. Также предполагается, что она неотрицательная, иначе говоря $f(x) \geqslant 0 \text{ }\forall x \in D$. 
	
	В рамках алгоритма генами будут называться вектора переменных, то есть точки $x \in D$. Хромосомами будут называться компоненты данного вектора. 
	
	Идея алгоритма состоит в создании некоторой популяции генов, которая будет постепенно эволюционировать и находить все лучшее приближение минимума $f(x)$. Размер популяции $M$, количество итераций $L$ и заменяемых генов $M_c$ задается исследователем.  Опишем основные шаги рассматриваемой вариации генетического алгоритма:
	
	\begin{enumerate}
		\item Генерация популяции из $M$ генов случайным образом.
		\item Подсчет функции приспособленности гена $fit(x) = \frac{1}{1 + f(x)}$ и сортировка популяции по убыванию значений приспособленности.
		\item Замена каждой особи с номером $m \in [2, M - M_c]$ на результат ее скрещивания со случайной особью в пуле. 
		
		Скрещивание двух генов $x, y \in \mathbb{R}^n$ подразумевает создание нового гена $z$, где каждая хромосома $z_i$ будет взята случайно из множества $\{x_i, y_i\}$, причем $\mathbb{P}(z_i = x_i) = \frac{fit(x)}{fit(x) + fit(y)}$.
		
		\item Замена $M_c$ наименее приспособленных особей случайно сгенерированными генами.
		\item Возврат к шагу 2, если не выполнен критерий останова (например, по достижению максимального числа итераций $L$).
		
	\end{enumerate}

	Заметим, что лучшая особь на каждой итерации не изменяется, поэтому алгоритм гарантированно не ухудшает уже найденное приближение. Иначе говоря, последовательность значений функции будет монотонно(нестрого) убывать. Если $f(x)$ -- ограничена на множестве $D$, то по теореме Вейерштрасса вышеописанная последовательность будет иметь предел при устремлении количества итераций к бесконечности.
	
	
	\subsection{Минимизация числовой функции}
	
	Пусть 
	\[f(x) = \sum_{k=1}^{n}x_k^2 (1 + |\sin{(100x_k)}|)\]
	
	Требуется минимизировать данную функцию на множестве $D = [-10, 10]^n$. Нетрудно видеть, что глобальным минимумом является нулевой вектор размера $n$. Однако градиентными методами данную задачу решить представляется очень сложным ввиду большого количества локальных минимумов. Подтверждением этого тезиса является график функции для $n=1$.
	
	
	\begin{figure}[H]
		\includegraphics[width = \textwidth]{gen_1.png}{}
		\caption{График функции $f(x) = x^2 (1 + |\sin(100x)|)$}
		\label{fig: gen_alg_test_1}
	\end{figure}
	
	
	
	
	Рассмотрим случай $n=5$. Сначала реализуем расчет значений функции $f(x)$ в MATLAB: 
	
	\begin{lstlisting}[
		frame=single,
		numbers=left,
		style=Matlab-Pyglike]
		function [y] = gen_alg_test_1(X)
		dim = size(X, 2);
		y = 0;
		for i = 1:dim
		  y = y + X(:, i).^2 .* (1 + abs(sin(100 * X(:, i))));
		end		
	\end{lstlisting}

	Далее напишем код генетического алгоритма в соответствии с вышеописанными шагами для случая $n=5$. Генерация гена будет происходить из равномерного распределения на множестве $[-10, 10]^5$.
	
	\begin{lstlisting}[
		frame=single,
		numbers=left,
		style=Matlab-Pyglike]
		function [y_min, x_min] = gen_alg_unif(f, D, M, M_c, L) 
		dim = size(D, 1);
		X = zeros(M, dim);
		for i = 1:dim
		  x_lower = D(i, 1);
		  x_upper = D(i, 2);
		  X(:, i) = unifrnd(x_lower, x_upper, M, 1);
		end
		for t = 1:L
		  fit = ones(M, 1) ./ (1 + f(X));
		  X = [fit X];
		  X = sortrows(X, 1, "descend");
		  fit = X(:, 1);
		  X = X(:, 2:(dim + 1));
		  male_idx = 2:M - M_c;
		  female_idx = randsample(1:length(M), M - M_c - 1, true);
		  male = X(male_idx, :);
		  female = X(female_idx, :);
		  child_male_gen_ind = zeros(M - M_c - 1, dim);
		  male_proba = fit(male_idx) / (fit(male_idx) + fit(female_idx));
		  for i = 1:M - M_c - 1
			  child_male_gen_ind(i, :) = binornd(1, male_proba(i), 1, dim);
		  end
		  X = male .* child_male_gen_ind + female .* (1 - child_male_gen_ind);
		  replace_ind = (M - M_c + 1):M;
		  for i = 1:dim
			  x_lower = D(i, 1);
			  x_upper = D(i, 2);
			  X(replace_ind, i) = unifrnd(x_lower, x_upper, M_c, 1);
		  end
		end
		x_min = X(1, :);
		y_min = f(x_min);
	\end{lstlisting}
	

	Зафиксируем параметры алгоритма: $M = 1000, M_c = 200, L = 100$. В результате запуска получим:
	
	\begin{lstlisting}[
		frame=single,
		numbers=left,
		style=Matlab-Pyglike]
		>> [y_min, x_min] = gen_alg_unif(@gen_alg_test_1, repmat([-10, 10], 5, 1), 1000, 300, 100)
		
		y_min =
		
		0
		
		
		x_min =
		
		0     0     0     0     0	
	\end{lstlisting}

	Видно, что алгоритм нашел глобальный минимум, причем время исполнения не превышает одной секунды.
	
	\subsection{Дискретная задача}
	
	Перейдем к следующей задаче -- имеется множество $A = \{1, 2, \dots, n\}$, которое необходимо разбить на два непересекающихся множества $K_1$ и $K_2$ так, чтобы 
	
	\[H(K_1, K_2) =  \left|\sum_{a_k \in K_1}a_k - \sum_{a_m \in K_2}a_m \right|\]
	
	была минимальна. Причем $K_1 \cup K_2 = A, K_1 \cap K_2 = \varnothing$.
	
	Известно, что данная задача не решается точно за полиномиальное время -- воспользуемся генетическим алгоритмом для нахождения приближенного решения.
	
	Заметим, что разбиение $A$ взаимно-однозначно задается характеристическим вектором подмножества $K_1$, то есть вектором $x = \{\mathbb{I}\{i \in K_1\}\}_{i=1}^n$. Тогда вышеописанная задача сводится к минимизации  следующей функции 
	\[\begin{aligned}
		f(x) &= \left|\sum_{k=1}^{n}k \cdot \mathbb{I}\{x_k = 1\}  - \sum_{k=1}^{n}k \cdot \mathbb{I}\{x_k = 0\} \right| = \left|\sum_{k=1}^{n}k \cdot (\mathbb{I}\{x_k = 1\}  -\mathbb{I}\{x_k = 0\})\right|, \\
		&\text{где }x \in \{0, 1\}^n \\
	\end{aligned}\]
	
	Решим эту задачу, используя MATLAB. Реализация функции $f(x)$ будет иметь вид:
	
	\begin{lstlisting}[
		frame=single,
		numbers=left,
		style=Matlab-Pyglike]
		function [y] = gen_alg_test_2(X)
		dim = size(X, 2);
		start = 1;
		stop = start + dim - 1;
		A = repmat(start:stop, size(X, 1), 1);
		
		Z = zeros(size(X, 1), dim);
		mask_0 = X == 0;
		Z(mask_0) = A(mask_0);
		sum_0 = sum(Z, 2);
		
		Z = zeros(size(X, 1), dim);
		mask_1 = X == 1;
		Z(mask_1) = A(mask_1);
		sum_1 = sum(Z, 2);
		
		y = abs(sum_1 - sum_0);	
	\end{lstlisting}

	Генетический алгоритм будет применяться аналогично предыдущей задаче, однако теперь для создания случайного гена каждая хромосома будет независимо генерироваться из распределения Бернулли с параметром $p = \frac{1}{2}$. Тогда код генетического алгоритма будет следующим:
	
		\begin{lstlisting}[
		frame=single,
		numbers=left,
		style=Matlab-Pyglike]
		function [y_min, x_min] = gen_alg_bern(f, dim, M, M_c, L) 
		X = zeros(M, dim);
		X = binornd(1, 0.5, M, dim);
		for t = 1:L
		  fit = ones(M, 1) ./ (1 + f(X));
		  X = [fit X];
	 	 X = sortrows(X, 1, "descend");
		  fit = X(:, 1);
		  X = X(:, 2:(dim + 1));
		  male_idx = 2:M - M_c;
		  female_idx = randsample(1:length(M), M - M_c - 1, true);
		  male = X(male_idx, :);
		  female = X(female_idx, :);
		  child_male_gen_ind = zeros(M - M_c - 1, dim);
		  male_proba = fit(male_idx) / (fit(male_idx) + fit(female_idx));
		  for i = 1:M - M_c - 1
		    child_male_gen_ind(i, :) = binornd(1, male_proba(i), 1, dim);
		  end
		  X = male .* child_male_gen_ind + female .* (1 - child_male_gen_ind);
		  replace_ind = (M - M_c + 1):M;
		  X(replace_ind, :) = binornd(1, 0.5, M_c, dim);
		end
		x_min = X(1, :);
		y_min = f(x_min);
	\end{lstlisting}

	Оставим параметры алгоритма фиксированными на прежнем уровне и запустим оптимизацию для $n=100, n=1000, n=10000$.
	
	 	\begin{lstlisting}[
	 	frame=single,
	 	numbers=left,
	 	style=Matlab-Pyglike]
	 	>> [y_min, ~] = gen_alg_bern(@gen_alg_test_2, 100, 1000, 300, 100)
	 	
	 	y_min =
	 	
	 	0
	 	
	 	>> [y_min, ~] = gen_alg_bern(@gen_alg_test_2, 1000, 1000, 300, 100)
	 	
	 	y_min =
	 	
	 	0
	 	
	 	
	 	>> [y_min, ~] = gen_alg_bern(@gen_alg_test_2, 10000, 1000, 300, 100)
	 	
	 	y_min =
	 	
	 	22
	 	
	 \end{lstlisting}
 
 	Заметим, что при $n=100$ и $n=1000$ генетический алгоритм нашел оптимальное разбиение исходного множества на два подмножества с одинаковой суммой. При $n=10000$ суммы уже получились не одинаковыми, однако значение целевой функции оказалось относительно близким к нулю.
 	
 	\subsection{Линейное программирование}
 	
 	Задача линейного программирования заключается в минимизации линейной функции при условии ограничений-равенств и ограничений-неравенств, где функции-ограничения имеют линейный вид. Кроме того, на некоторые переменные может быть наложено ограничение целочисленности и нахождение на некотором отрезке. Таким образом, данная задача будет иметь следующий вид
 	
 	\[\left\{\begin{aligned}
 		f(x) &= c^T x \rightarrow \underset{x \in \mathbb{R}^n}{\min} \\ 
 		Ax &\leqslant b \\
 		A_{eq} x &= b_{eq} \\
 		lb_i &\leqslant x_i \leqslant ub_i \text{, } i \in J \\
 		x_i &\in \mathbb{Z}  \text{, }  i \in Z\\
 	\end{aligned}\right.,\]
 	где $A, A_{eq} \in Mat_{m \text{ } n}(\mathbb{R})$; $b, b_{eq} \in \mathbb{R}^m$; $c \in \mathbb{R}^n$; $J, Z \subseteq \{1, 2, \dots, n\}$;
 	 
 	 $lb_i, ub_i \in \mathbb{R} \text{, } i \in J$
 	
 	Теперь от данной общей постановки перейдем к более узкой, где нет ограничений-равенств и верхних границ значений переменных. Также будем рассматривать только неотрицательные значения переменных и параметров задачи. Кроме того, теперь задача будет заключаться в максимизации целевой функции. То есть
 	
 	\[\left\{\begin{aligned}
 		f(x) &= c^T x \rightarrow \underset{x \in \mathbb{R}^n}{\max} \\ 
 		Ax &\leqslant b \\
		x_i &\geqslant 0  \text{, } i \in  \{1, 2, \dots, n\} \\
 		x_i &\in \mathbb{Z}  \text{, }  i \in Z\\
 	\end{aligned}\right.,\]
 	где $A \in Mat_{m \text{ } n}(\mathbb{R}_+)$; $b \in \mathbb{R}_+^m$; $c \in \mathbb{R}_+^n$; $Z \subseteq \{1, 2, \dots, n\}$;
 	
 	Данная задача чаще всего решается симплекс-методом, однако генетический алгоритм к ней также применим -- реализуем его в MATLAB.
 	
 	Заметим, что основная часть этого алгоритма уже была реализована в вышеописанных задачах, поэтому необходимо лишь адаптировать его под задачу линейного программирования. В частности, необходимо поменять генерацию генов, а также подсчет функции приспособленности.
 	
 	В силу того что все переменные и параметры неотрицательны, для каждой переменной может быть вычислен отрезок, в который она точно должна попасть в соответствии с ограничениями-неравенствами. Легко видеть, что
 	 \[x_j \in [0, v_j] \text{ , где } v_j = \underset{i: a_{ij} \neq 0}{\min{\frac{b_i}{a_{ij}}}} \]
 	 
 	 Последнее неравенство получается, если обнулить значения всех переменных, кроме той, что имеет номер $j$, далее решить систему $Ax \leqslant b$. В силу вышеупомянутой неотрицательности всех параметров и переменных полученное неравенство обязательно соблюдается для допустимых точек.
 	  
 	 Тогда каждую вещественнозначную хромосому для $j$ - ой компоненты гена можно генерировать из равномерного распределения на отрезке $[0, v_j]$, а каждую целочисленную - на множестве $\{0, 1, \dots \lfloor v_j \rfloor \}$. Если правая граница равна нулю, то соответствующая переменная всегда будет равна нулю.
 	 
 	 Однако стоит отметить, что вышеописанное условие является лишь необходимым, но не достаточным -- оно гарантирует неотрицаетльность всех переменных и целочисленность соответствующих компонент гена, однако условие $Ax\leqslant b$ нужно проверять дополнительно. Чтобы учесть это в рамках генетического алгоритма, присвоим нулевые значения приспособленности тем генам, для которых не выполняется $Ax \leqslant b$. Для остальных особей приспособленность будет считаться как
 	 \[fit(x) = c^T x\] 
 	 
 	 Тогда код генетического алгоритма будет иметь следующий вид
 	 
 	 
 	 \begin{lstlisting}[
 	 	frame=single,
 	 	numbers=left,
 	 	style=Matlab-Pyglike]
 	 	function [y_best, x_best] = gen_alg_lp(A, c, b, int_idx, M, M_c, L) 
 	 	% get continious variables indexes
 	 	n = size(A, 2);
 	 	cont_idx = 1:n;
 	 	cont_idx(int_idx) = [];
 	 	% determine bounds of variables
 	 	B = repmat(b, 1, n);
 	 	x_max = min(B ./ (A + 1e-8))';
 	 	x_min = zeros(n, 1);
 	 	
 	 	% generate population
 	 	X = zeros(M, n);
 	 	% continious variables
 	 	for i = cont_idx
 	 	  x_lower = x_min(i);
 	 	  x_upper = x_max(i);
 	 	  X(:, i) = unifrnd(x_lower, x_upper, M, 1);
 	 	end
 	 	% integer variables
 	 	for i = int_idx
 	 	  x_lower = x_min(i);
 	 	  x_upper = floor(x_max(i)) + 1;
 	   	  X(:, i) = randi([x_lower, x_upper], M, 1);
 	 	end
 	 	
 	 	% main loop
 	 	for t = 1:L
 	 	  % profit calc
 	 	  y = X * c;
 	 	  % fit func calc and checking constraint
 	 	  res_usage = A * X';
 	 	  res_cond_mask = all(res_usage <= repmat(b, 1, M), 1);
 	 	  fit = y;
 	 	  fit(~res_cond_mask) = 0;
 	 	  % sort gens
 	 	  X = [fit X];
 	 	  X = sortrows(X, 1, "descend");
 	 	  fit = X(:, 1);
 	 	  X = X(:, 2:(n + 1));
 	 	  % crossing
 	 	  male_idx = 2:M - M_c;
 	 	  female_idx = randsample(1:length(M), M - M_c - 1, true);
 	 	  male = X(male_idx, :);
 	 	  female = X(female_idx, :);
 	 	  child_male_gen_ind = zeros(M - M_c - 1, n);
 	 	  male_proba = fit(male_idx) ./ (fit(male_idx) + fit(female_idx) + 1e-8);
 	 	  for i = 1:M - M_c - 1
 	 	    child_male_gen_ind(i, :) = binornd(1, male_proba(i), 1, n);
 	 	  end
 	 	  X = male .* child_male_gen_ind + female .* (1 - child_male_gen_ind);
 	 	  replace_ind = (M - M_c + 1):M;
 	 	  % replacing M_c gens
 	 	  % cont variables
 	 	  for i = cont_idx
 	 	    x_lower = x_min(i);
 	 	    x_upper = x_max(i);
 	 	    X(replace_ind, i) = unifrnd(x_lower, x_upper, M_c, 1);
 	 	  end
 	 	  % integer variables
 	 	  for i = int_idx
 	 	    x_lower = x_min(i);
 	 	    x_upper = floor(x_max(i)) + 1;
 	 	    X(replace_ind, i) = randi([x_lower, x_upper], M_c, 1);
 	 	  end
 	 	end
 	 	x_best = X(1, :);
 	 	y_best = x_best * c;
 	 	
 	 \end{lstlisting}
 	
	Теперь протестируем данный алгоритм на конкретной задаче и сравним полученный результат с симплекс-методом.
	
	Пусть 
	\[\begin{aligned}
		A &= \begin{pmatrix}
			10 & 20& 30 \\ 
			60 & 20& 10 \\ 
			35.4 & 40.2& 50 \\ 
		\end{pmatrix}
	\end{aligned} \\ 
		b = \begin{pmatrix}
		1000 \\
		750 \\
		830.5
	\end{pmatrix} \\
		c = \begin{pmatrix}
		30 \\
		15 \\ 
		22.5 \\
	\end{pmatrix} \\
	x_2, x_3 \in \mathbb{Z}\]
	
	Параметры алгоритма возьмем следующими: $M=1000, M_c = 200, L=300$. Тогда выполним максимизацию исходной функции с помощью генетического алгоритма и встроенного в MATLAB симплекс-метода.
	
	\begin{lstlisting}[
		frame=single,
		numbers=left,
		style=Matlab-Pyglike]
		>> M = 1000;
		M_c = 200;
		L = 300;
		>> A = [[10, 20, 30]; [60, 20, 10]; [35.4, 40.2, 50]];
		b = [1000, 750, 830.5]';
		c = [30, 15, 22.5]';
		int_idx = [2, 3];
		>> simp_meth = intlinprog(-c, int_idx, A, b ,[], [] , zeros(3, 1))
		LP:                Optimal objective value is -528.968254.                                          
		
		Heuristics:        Found 2 solutions using ZI round.                                                
		Upper bound is -511.009070.                                                      
		Relative gap is 2.72%.                                                          
		
		Cut Generation:    Applied 1 mir cut.                                                               
		Lower bound is -524.957627.                                                      
		Relative gap is 0.00%.                                                          
		
		
		Optimal solution found.
		
		Intlinprog stopped at the root node because the objective value is within a gap tolerance of the optimal value,
		options.AbsoluteGapTolerance = 0 (the default value). The intcon variables are integer within tolerance,
		options.IntegerTolerance = 1e-05 (the default value).
		
		
		simp_meth =
		
		10.7486
		0
		9.0000
		
		>> [~, x_best] = gen_alg_lp(A, c, b, int_idx, M, M_c, L)
		
		x_best =
		
		10.7450         0    9.0000
		
	\end{lstlisting}
	
	Видно, что найденные приближения решения задачи линейного программирования, построенные симплекс-методом и генетическим алгоритмом, равны с точностью до 3 знака после запятой, что свидетельствует о том, что генетический алгоритм успешно справился с данной задачей. Причем его время работы не превышает 1-2 секунд.
	
	
	
	
	
\end{document} % конец документа