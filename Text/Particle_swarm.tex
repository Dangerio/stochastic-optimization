% Этот шаблон документа разработан в 2014 году
% Данилом Фёдоровых (danil@fedorovykh.ru) 
% для использования в курсе 
% <<Документы и презентации в \LaTeX>>, записанном НИУ ВШЭ
% для Coursera.org: http://coursera.org/course/latex .
% Исходная версия шаблона --- 
% https://www.writelatex.com/coursera/latex/2

\documentclass[../main.tex]{subfiles}

%%% Работа с русским языком
\usepackage{cmap}					% поиск в PDF
\usepackage{mathtext} 				% русские буквы в формулах
\usepackage[T2A]{fontenc}			% кодировка
\usepackage[utf8]{inputenc}			% кодировка исходного текста
\usepackage[english,russian]{babel}	% локализация и переносы

%%% Дополнительная работа с математикой
\usepackage{amsfonts,amssymb,amsthm,mathtools} % AMS
\usepackage{amsmath}
\usepackage{icomma} % "Умная" запятая: $0,2$ --- число, $0, 2$ --- перечисление

%% Номера формул
%\mathtoolsset{showonlyrefs=true} % Показывать номера только у тех формул, на которые есть \eqref{} в тексте.
%% Шрифты
\usepackage{euscript}	 % Шрифт Евклид
\usepackage{mathrsfs} % Красивый матшрифт

%% Свои команды
\DeclareMathOperator{\sgn}{\mathop{sgn}}

%% Перенос знаков в формулах (по Львовскому)
\newcommand*{\hm}[1]{#1\nobreak\discretionary{}
	{\hbox{$\mathsurround=0pt #1$}}{}}

%%% Работа с картинками
\usepackage{graphicx}  % Для вставки рисунков
\graphicspath{{images/}{images2/}}  % папки с картинками
\setlength\fboxsep{3pt} % Отступ рамки \fbox{} от рисунка
\setlength\fboxrule{1pt} % Толщина линий рамки \fbox{}
\usepackage{wrapfig} % Обтекание рисунков и таблиц текстом

%%% Работа с таблицами
\usepackage{array,tabularx,tabulary,booktabs} % Дополнительная работа с таблицами
\usepackage{longtable}  % Длинные таблицы
\usepackage{multirow} % Слияние строк в таблице


\usepackage{indentfirst}
\usepackage{hyperref}
\usepackage{booktabs}
\usepackage{float}
\usepackage[table]{xcolor}

% код в matlab
\usepackage{matlab-prettifier}
\usepackage{listings,lstautogobble}
\lstset{autogobble=true}



\begin{document} % конец преамбулы, начало документа
	В данной главе будет использован метод роения частиц для минимизации нескольких трудных для оптимизации функций.
	
	Для начала реализуем метод роения частиц, причем для ускорения работы алгоритма постараемся избежать использования циклов там, где это возможно. Вместо этого будем стараться выполнять операции в матричном виде.
	
	Также стоит отметить, что начальное приближение вектора положения частиц и их скоростей будет генерироваться из равномерного распределения на отрезке $[\underline{x}, \bar{x}]$ и $[\underline{x} - \bar{x}, \bar{x} - \underline{x}]$ соответственно. Параметры $\underline{x},  \bar{x}$ будут задаваться исследователем.
	
	В таком случае код метода роения частиц будет иметь следующий вид:
	
	\begin{lstlisting}[
		frame=single,
		numbers=left,
		style=Matlab-Pyglike]
		function [argvalue_global, argmin_global] = particle_swarm(f, dim, x_lower, x_upper, alpha, beta, gamma, M, L) 
		X = unifrnd(x_lower, x_upper, M, dim);
		V = unifrnd(x_lower - x_upper, x_upper - x_lower, M, dim); 
		X = X + V;
		y = f(X);
		argvalue_local = y;
		argmin_local = X;
		
		[argvalue_global, idxmin_global] = min(argvalue_local);
		argmin_global = argmin_local(idxmin_global, :);
		
		for t = 1:L
		  xi_1 = repmat(unifrnd(0, 1, M, 1), 1, dim);
		  xi_2 = repmat(unifrnd(0, 1, M, 1), 1, dim);
	 	  V = alpha * V + beta * xi_1 .* (argmin_local - X) +  gamma * xi_2 .* (argmin_global - X);
		  X = X + V;
		  y = f(X);
		  [argvalue_local, idxmin_local] = min([argvalue_local, y], [], 2);
		  mask_idx = find(idxmin_local == 2);
		  argmin_local(mask_idx, :) = X(mask_idx, :);
		  [argvalue_global, idxmin_global] = min(argvalue_local);
		  argmin_global = argmin_local(idxmin_global, :);
		end
	\end{lstlisting}
	
	Протестируем данный метод на функции Розенброка, которая имеет следующий вид:
	
	\[f(x, y) = (1 - x)^2 + 100(y-x^2)^2\]
	
	Видно, что глобальный минимум функции достигается в точке $(x, y) = (1, 1)$
	
	Положим $\underline{x} = -10,  \bar{x} = 10, \alpha = 0.95, \beta = 0.2, \gamma=0.2$, а также количество частиц зафиксируем на уровне 100, количество итераций будет равным 1000. В результате применения метода получим следующее:
	
	\begin{lstlisting}[
		frame=single,
		numbers=left,
		style=Matlab-Pyglike]
		>> F = @(X)(1 - X(:, 1)).^2 + 100 * (X(:, 2) - X(:, 1).^2).^2
		
		F =
		
		function_handle with value:
		
		@(X)(1-X(:,1)).^2+100*(X(:,2)-X(:,1).^2).^2
		
		>> [y, x] = particle_swarm(F, 2, -10, 10, 0.95, 0.2, 0.2, 100, 1000)
		
		y =
		
		5.3524e-22
		
		
		x =
		
		1.0000    1.0000
		
		
	\end{lstlisting}
	
	Таким образом, глобальный минимум был найден, причем алгоритм отработал очень быстро, чему поспособствовало использование матричных вычислений.
	
	Теперь протестируем метод роения частиц на другой трудной для оптимизации функции, имеющей следующий вид:
	
	\[\begin{aligned}
		f(x, y, z) &= 0.01(x-0.5)^2 + |x^2-y| + |x^2 - z| \\
		-0.2 &\leqslant x \leqslant 1, \text{ } -0.3 \leqslant y \leqslant 1, \text{ }
		 -0.5 \leqslant z \leqslant 1 \\
	\end{aligned}\]

	Известно, что глобальный минимум этой функции достигается в точке $(x, y, z) = (0.5, 0.25, 0.25)$
	
	Применим метод роения частиц к этой функции, пусть $\underline{x} = -1,  \bar{x} = 1$, а количество итераций возьмем равным $10^6$ (оно было подобрано на основании нескольких запусков). В результате получим следующее:
	
	\begin{lstlisting}[
		frame=single,
		numbers=left,
		style=Matlab-Pyglike]
	
	>> F = @(X)0.01 * (X(:, 1) - 0.5).^2 + abs(X(:, 1).^2 - X(:, 2)) + abs(X(:, 1).^2 - X(:, 3))
	
	F =
	
	function_handle with value:
	
	@(X)0.01*(X(:,1)-0.5).^2+abs(X(:,1).^2-X(:,2))+abs(X(:,1).^2-X(:,3))
	
	>> [y, x] = particle_swarm(F, 3, -1, 1, 0.95, 0.2, 0.2, 100, 10^6)
	
	y =
	
	0
	
	
	x =
	
	0.5000    0.2500    0.2500
	
	>> 
	
		
		
	\end{lstlisting}

	Видно, что глобальный минимум был найден. Алгоритм работал примерно 10-12 секунд, что можно считать неплохим результатом.
	
	Таким образом, в данной главе был реализован метод роения частиц в MATLAB и с помощью него были успешно найдены глобальные минимумы двух непростых для оптимизации функций.
\end{document} % конец документа